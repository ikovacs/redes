\section{Conclusiones}

A lo del desarrollo del trabajo práctico, se pudo notar que la cantidad los paquetes ARP transmitidos en un cierto tiempo, es siempre menor a la cantidad de paquetes de otros tipos. Eso es esperado, debido a la funcionalidad que presenta.

Por otro lado, nos dimos cuenta que es necesario realizar distintos enfoques para el estudio de la información obtenida para cada una de las fuentes. Por ejemplo, a la hora de detectar los nodos distinguidos dentro de una red, con la ayuda de dos tipos de análisis dintintos se pudo difenrenciar cuales eran éstos. Con la observación de unos de los gráficos obtenidos se intuyeron cuales eran los nodos de la red correspondiente. Luego, con la comparación con la entropía promedio, se pudieron corroborar hipótesis ciertas o descartar casos erróneos.