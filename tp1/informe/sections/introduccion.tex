\section{Introducción}
EL objetivo principal del siguiente trabajo práctico consiste en observar distintas redes, mediante diferentes herramientas, y realizar un exhaustivo análisis a partir de la información obtenida. Las herramientas utilizadas para poder escuchar y manipular los paquetes de la red son: Wireshark y Scapy. Una vez recolectada la información requerida de los paquetes de la red, se realizarán comparaciones y conclusiones a partir de lo observado. 

En la primera parte del trabajo práctico se definirá una fuente diferente a la dada por la cátedra, se calculará las probabilidades de las fuentes S y S1 y en consecuencia, sus entropías. 

Luego, en la segunda mitad, se realizará un análisis que permita, para cada una de las redes estudiadas anteriormente, definir los protocolos distinguidos, analizar el overhead impuesto por el protocolo ARP y, además, determinar los nodos distinguidos. 

