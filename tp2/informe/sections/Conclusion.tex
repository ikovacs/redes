\section{Conclusión}

A lo largo del trabajo práctico, se fueron realizando diversos experimentos para poder comprender el comportamiento de la red mundial, por 
donde se transmiten millones de paquetes todo el tiempo. Se pudo observar que un mensaje enviado desde un host a otro, realiza 
un recorrido inmenso, saltando por diversos host intermedios, los cuales no son siempre los mismos. Cada host intermedio, presenta un comportamiento completamente diferente y hace al aumento del tiempo en el que llegará un paquete a destino. 

En este trabajo práctico se observaron muchos router que no tenían habilitado un sistema de respuesta, por lo que presentó dificultades al momento de realizar los análisis necesarios. Estos hops intermedios fueron ignorados completamente, para tratar de afectar lo menos posible al RTT promedio. Además, se pudo notar, que dependiendo del día o del momento, en el cual se realizan las mediciones, el tiempo total obtenido es notablemente diferente. 

Creemos que se podrían haber realizado análisis más profundos, quizás tomando medicientes cada diez 
minutos durante todo un día, pero probablemente los RTT para cada hop intermedio no se diferenciarían mucho de los actuales. 

También hemos notado que a mayor cantidad de paquetes mejores resultados para los análisis futuros, y que a veces es necesario realizar 
los experimentos más de una vez para llegar a una buena conclusión de las muestras obtenidas.