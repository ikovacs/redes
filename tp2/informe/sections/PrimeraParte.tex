\section{Caracterizando rutas}

\subsection{Estimación del RTT}
A partir de la tool implementada que permite realizar un \textbf{traceroute}, se pueden obtener los RTT entre cada salto de hops para una ruta a una IP determinada. De esta manera, se plantea en primer lugar 4 universidades diferentes, que serán utilizadas para poder llevar a cabo el análisis necesario.

\begin{itemize}
 \item {\bf Universidad Humboldt de Berlín}

	{\bf Distancia}: 11.910 km

	{\bf IP}: 141.20.5.188 (\url{www.hu-berlin.de}{})

 \item {\bf Universidad de Moscú}

	{\bf Distancia}: 13.484 km

	{\bf IP}: 188.44.33.1 (\url{http://www.msu.ru/}{})

 \item {\bf Universidad de Tokio}

	{\bf Distancia}: 18.360 km

	{\bf IP}: 210.152.135.178 (\url{www.u-tokyo.ac.jp}{})

 \item {\bf Universidad de Boston}

	{\bf Distancia}: 8.658 km

	{\bf IP}: 54.230.225.44 (\url{www.bu.edu}{})

\end{itemize}

Las distancias descriptas corresponden a la distancia lineal que hay entre la universidad y un punto en común en Buenos Aires. La misma se obtuvo utilizando una opción dada por Google Maps \footnote{\url{maps.google.com}{}}. Ese dato se podrá utilizar para poder obtener el RTT teórico y aproximado de un paquete para cada una de las universidad a analizar.

Asumiendo que los enlaces son siempre de fibra óptica, y que el velocidad de propagación de las señales es de $2 \times 10^{5}$ km/s podemos estimar el RTT de la siguiente manera:

\begin{itemize}
 \item Universidad Humboldt de Berlín:
\begin{equation}
 	RTT = 2 \times T_{prop} = 2 \times (Dist / V_{prop}) = 2 \times (11910 \text{ km} / 2\times10^5 \text{ km/s}) = 59.5517  \text{ ms}
\end{equation}

 \item Universidad de Moscú:
 \begin{equation}
	\textbf{cambiar el resultado de la cuenta!!}
 	RTT = 2 \times T_{prop} = 2 \times (Dist / V_{prop}) = 2 \times (13484 \text{ km} / 2\times10^5 \text{ km/s}) = 40.655 \text{ ms}
 \end{equation}

 \item Universidad de Tokio:
 \begin{equation}
 	RTT = 2 \times T_{prop} = 2 \times (Dist / V_{prop}) = 2 \times (18360  \text{ km} / 2\times10^5 \text{ km/s}) = 91.8 \text{ ms}
 \end{equation}

 \item Universidad de Boston:
 \begin{equation}
 	RTT = 2 \times T_{prop} = 2 \times (Dist / V_{prop}) = 2 \times (8658 \text{ km} / 2\times10^5 \text{ km/s}) = 43.29 \text{ ms}
 \end{equation}

\end{itemize}

El valor obtenido representa un cota inferior muy burda del tiempo de comunicación entre Buenos Aires y las universidades seleccionadas. Para poder probarlo se ejecuta el traceroute desarrollado por nosotros utilizando Scapy. El traceroute envía DECIR LA CANT DE PAQUETES a cada hop intermedio, para obtener luego el tiempo promedio del RTT. A continuación, se presenta en gráficos la diferencia entre los RTT obtenidos en la práctica contra los calculados:

%\centerline{\includegraphics[width=0.8\textwidth]{.graficos/traceroute_empirico_alemania.png}%{Alemania: Traceroute vs Cota teérica}{0.5}{tr_empirico_canada}

%\centerline{\includegraphics[width=0.8\textwidth]{.graficos/traceroute_empirico_moscu.png}%{Moscú: Traceroute vs Cota teérica}{0.5}{tr_empirico_rusia}

%\centerline{\includegraphics[width=0.8\textwidth]{.graficos/traceroute_empirico_japon.png}%{Tokio: Traceroute vs Cota teórica}{0.5}{tr_empirico_china}

%\centerline{\includegraphics[width=0.8\textwidth]{.graficos/traceroute_empirico_eeuu.png}%{Estados Unidos: Traceroute vs Cota teórica}{0.5}{tr_empirico_china}
