\section{Caracterizando rutas}

\subsection{Estimación del RTT}
A partir de la tool implementada, que permite realizar un \textbf{traceroute}, se pueden obtener los RTT entre cada salto de hops para una ruta a una IP determinada. De esta manera, se plantea en primer lugar 4 universidades diferentes, que serán utilizadas para poder llevar a cabo el análisis necesario utilizando la herramienta.

\begin{itemize}
 \item {\bf Universidad Humboldt de Berlín}

	{\bf Distancia}: 11910 km

	{\bf IP}: 141.20.5.188 (\url{www.hu-berlin.de}{})

 \item {\bf Universidad de Moscú}

	{\bf Distancia}: 13484 km

	{\bf IP}: 188.44.33.1 (\url{www.msu.ru}{})

 \item {\bf Universidad de Boston}

	{\bf Distancia}: 8680 km

	{\bf IP}: 54.230.225.44 (\url{www.bu.edu}{})

 \item {\bf Universidad de Tokio}

	{\bf Distancia}: 18360 km

	{\bf IP}: 210.152.135.178 (\url{www.u-tokyo.ac.jp}{})

\end{itemize}

Las distancias descriptas corresponden a la distancia lineal que hay entre la universidad y un punto en común en Buenos Aires. La misma se obtuvo utilizando una opción dada por Google Maps \footnote{\url{maps.google.com}{}}. Ese dato se podrá utilizar para poder obtener el RTT teórico y aproximado de un paquete para cada una de las universidad a analizar.

Asumiendo que los enlaces son siempre de fibra óptica, y que el velocidad de propagación de las señales es de $2 \times 10^{5}$ km/s podemos estimar el RTT de la siguiente manera:

\begin{itemize}
 \item Universidad Humboldt de Berlín:
\begin{equation}
 	RTT = 2 \times T_{prop} = 2 \times (Dist / V_{prop}) = 2 \times (11910 \text{ km} / 2\times10^5 \text{ km/s}) = 119.1  \text{ ms}
\end{equation}

 \item Universidad de Moscú:
 \begin{equation}
 	RTT = 2 \times T_{prop} = 2 \times (Dist / V_{prop}) = 2 \times (13484 \text{ km} / 2\times10^5 \text{ km/s}) = 134.84 \text{ ms}
 \end{equation}

 \item Universidad de Boston:
 \begin{equation}
 	RTT = 2 \times T_{prop} = 2 \times (Dist / V_{prop}) = 2 \times (8680 \text{ km} / 2\times10^5 \text{ km/s}) = 86.8 \text{ ms}
 \end{equation}

 \item Universidad de Tokio:
 \begin{equation}
 	RTT = 2 \times T_{prop} = 2 \times (Dist / V_{prop}) = 2 \times (18360  \text{ km} / 2\times10^5 \text{ km/s}) = 183.6 \text{ ms}
 \end{equation}

\end{itemize}

El valor obtenido representa un cota inferior muy burda del tiempo de comunicación entre Buenos Aires y las universidades seleccionadas. Esto se debe a que no contempla los obstáculos que se presentan desde que un paquete sale del hop inicial y llega a destino. Hay ciertas situaciones que fomentan un aumento importante del tiempo de RTT del mensaje. Se presenta a continuación, las diferencias entre el RTT teórico y el obtenido utilizando una herramienta brindada por el sistema operativo \footnote{\url{http://ping.eu}{}}.

\centerline{\includegraphics[width=0.8\textwidth]{imagenes/1Parte-comparacionRTT}}

Se muestran los resultado utilizando escala logarítmica para poder observarlos de manera clara. A lo largo del trabajo práctico se podrán ir observando las situaciones que hacen a que el RTT teórico sea tanto menor que el real.

Por otro lado, se decide comparar la herramienta desarrollada utilizando Scapy, contra el traceroute brindado por el sistema operativo. Ambos traceroutes envían una cierta cantidad de paquetes a cada host intermedio que se tiene para llegar al de destino, y miden el tiempo obtenido. Luego, se realiza un promedio del mismo. A continuación, se presentan los RTT obtenidos para cada host intermedio utilizando ambas herramientas. 

%\centerline{\includegraphics[width=0.8\textwidth]{.imagenes/traceroute_empirico_alemania.png}}

%\centerline{\includegraphics[width=0.8\textwidth]{.imagenes/traceroute_empirico_moscu.png}}

%\centerline{\includegraphics[width=0.8\textwidth]{.imagenes/traceroute_empirico_eeuu.png}}

%\centerline{\includegraphics[width=0.8\textwidth]{.imagenes/traceroute_empirico_japon.png}}

Conlusiones de eso..
Hablar de todos los casos posibles para que un host de mayor que otro..

Uno de los motivos que afecta a los tiempos en los que un paquete tarda en llegar a un host y luego retornar, es el del congestionamiento que puede existir en la red. El mismo surge a partir de un alto número de paquetes existentes en la misma, debido a un gran número de personas que intentan conectarse a un mismo lugar. Para poder observar este comportamiento, es decir, los fluctuantes valores de RTT para cada hop intermedio, se realizó el siguiente experimento, el cual consiste en analizar el tráfico de la red en distintas horas del día. Para ello se ejecutó el traceroute desarrollado en tres horarios diferentes. Se obtienen los siguientes gráficos:

%\centerline{\includegraphics[width=0.8\textwidth]{.imagenes/traceroute_empirico_alemania.png}}

%\centerline{\includegraphics[width=0.8\textwidth]{.imagenes/traceroute_empirico_moscu.png}}

%\centerline{\includegraphics[width=0.8\textwidth]{.imagenes/traceroute_empirico_eeuu.png}}

%\centerline{\includegraphics[width=0.8\textwidth]{.imagenes/traceroute_empirico_japon.png}}

Conclusiones de eso..

Luego, se agrega a la herramienta desarrollada por nosotros, un agregado para que se calcule el valor standard (ZRTT) de cada salto con respecto a la ruta global. El cálculo a realizarse es el siguiente:

 \begin{equation}
 	ZRTT_i = \frac{RTT_i - \overline{RTT}}{SRTT} 
 \end{equation}

 Siendo RTT$_{i}$ el RTT medido para el salto entre host número i, $\overline{RTT}$ el promedio para los RTT de todos los saltos realizados y por último el SRTT representa el desvío standard de los RTTs de la ruta, y se calcula de la siguiente manera:

\begin{equation}
 	SRTT = \sqrt{\frac{1}{n} \sum_{i=1}^{n} (RTT_i - \overline{RTT})^2}
 \end{equation}

El ZRTT nos permite definir de forma rápida y exacta cuánto se aleja un RTT de un salto específico respecto al promedio. Luego se podrán realizar diversos análisis y conclusiones utilizando los valores obtenidos en los siguientes gráficos. Estos últimos representan el ZRTT calculado para una medición hecha con mil paquetes para cada TTL con destino cada una de las universidades definidas previamente. 

%\centerline{\includegraphics[width=0.8\textwidth]{.imagenes/traceroute_empirico_alemania.png}}

%\centerline{\includegraphics[width=0.8\textwidth]{.imagenes/traceroute_empirico_moscu.png}}

%\centerline{\includegraphics[width=0.8\textwidth]{.imagenes/traceroute_empirico_eeuu.png}}

%\centerline{\includegraphics[width=0.8\textwidth]{.imagenes/traceroute_empirico_japon.png}}

