\section{Introducción}

A lo largo de este trabajo práctico se experimentará con técnicas
y herramientas utilizadas a nivel de red. Particularmente, se analizará
la herramienta \textit{traceroute} junto con los protocolos involucrados de esa capa. 

Para realizar los experimentos se eligieron cuatro universidades distintas: la Universidad Humboldt de Berlín, la Universidad de Moscú,
la Universidad de Tokio y la Universidad de Boston.

Se desarrollara y aplicará una herramienta que permitirá analizar el comportamiento de un paquete en la red. 

El trabajo práctico se divide en tres partes. La primera hace referencia a la herramienta de traceroute y la realización de cálculos a partir de los datos obtenidos. Luego, se plasmarán los mismos en diversos gráficos y diagramas para poder definir ciertos escenarios esperados. Por último, se aplicará otra herramienta para poder contrastar con la realidad cierto comportamiento teórico. 
